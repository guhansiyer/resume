\documentclass[letterpaper,11pt]{article}

\usepackage{latexsym}
\usepackage[empty]{fullpage}
\usepackage{titlesec}
\usepackage{marvosym}
\usepackage[usenames,dvipsnames]{color}
\usepackage{verbatim}
\usepackage{enumitem}
\usepackage[hidelinks]{hyperref}
\usepackage{fancyhdr}
\usepackage[english]{babel}
\usepackage{tabularx}
\usepackage{fontawesome5}
\usepackage{ulem}
\usepackage{soul}
\input{glyphtounicode}


%----------FONT OPTIONS----------
% sans-serif
% \usepackage[sfdefault]{FiraSans}
% \usepackage[sfdefault]{roboto}
% \usepackage[sfdefault]{noto-sans}
% \usepackage[default]{sourcesanspro}

% serif
% \usepackage{nunito}
% \usepackage{CormorantGaramond}
\usepackage{charter}

\pagestyle{fancy}
\fancyhf{} % clear all header and footer fields
\fancyfoot{}
\renewcommand{\headrulewidth}{0pt}
\renewcommand{\footrulewidth}{0pt}

% Adjust margins
\addtolength{\oddsidemargin}{-0.5in}
\addtolength{\evensidemargin}{-0.5in}
\addtolength{\textwidth}{1in}
\addtolength{\topmargin}{-.5in}
\addtolength{\textheight}{1.0in}

\urlstyle{same}

\raggedbottom
\raggedright
\setlength{\tabcolsep}{0in}

% Sections formatting
\titleformat{\section}{
  \vspace{-4pt}\scshape\raggedright\large
}{}{0em}{}[\color{black}\titlerule \vspace{-5pt}]

% Ensure that generate pdf is machine readable/ATS parsable
\pdfgentounicode=1

%-------------------------
% Custom commands
\newcommand{\resumeItem}[1]{
  \item\small{
    {#1 \vspace{-1.14pt}}
  }
}

\newcommand{\resumeSubheading}[4]{
  \vspace{-2pt}\item
    \begin{tabular*}{0.97\textwidth}[t]{l@{\extracolsep{\fill}}r}
      \textbf{#1} & #2 \\
      \vspace{0.25pt}
      \textit{\small#3} & \textit{\small #4} \\
    \end{tabular*}\vspace{-7pt}
}

\newcommand{\resumeSubSubheading}[2]{
    \item
    \begin{tabular*}{0.97\textwidth}{l@{\extracolsep{\fill}}r}
      \textit{\small#1} & \textit{\small #2} \\
    \end{tabular*}\vspace{-5pt}
}

\newcommand{\resumeProjectHeading}[2]{
    \item
    \begin{tabular*}{0.97\textwidth}{l@{\extracolsep{\fill}}r}
      \small#1 & #2 \\
    \end{tabular*}\vspace{-7pt}
}

\newcommand{\resumeSubItem}[1]{\resumeItem{#1}\vspace{-4pt}}

\renewcommand\labelitemii{$\vcenter{\hbox{\tiny$\bullet$}}$}

\newcommand{\resumeSubHeadingListStart}{\begin{itemize}[leftmargin=0.15in, label={}]}
\newcommand{\resumeSubHeadingListEnd}{\end{itemize}}
\newcommand{\resumeItemListStart}{\begin{itemize}}
\newcommand{\resumeItemListEnd}{\end{itemize}\vspace{-5pt}}

%-------------------------------------------
%%%%%%  RESUME STARTS HERE  %%%%%%%%%%%%%%%%%%%%%%%%%%%%


\begin{document}

%----------HEADING----------


\begin{tabular*}{\textwidth}{l@{\extracolsep{\fill}}r}
   \textbf{\huge{Guhan Iyer}} & Email: \href{mailto:guhan.iyer@uwaterloo.ca}{guhan.iyer@uwaterloo.ca}\\
    \href{https://www.linkedin.com/in/guhansiyer}{linkedin.com/in/guhansiyer \faIcon{external-link-alt}} \space $\vert$  \href{https://github.com/guhansiyer}{github.com/guhansiyer \faIcon{external-link-alt}} & Phone: +1 (226) 505-7658 \cr  
 \end{tabular*}

% \begin{center}
%     { \Huge \textbf{Guhan Iyer}} \\ \vspace{3pt}
%     \normalsize \raisebox{-0.1\height} ~
%     {\faPhoneSquare\ (226) 505-7658} {$\mid$} 
%     % \href{mailto:g2iyer@uwaterloo.ca}{\raisebox{-0.2\height}{\faEnvelope[regular]}\ \underline{guhan.iyer@uwaterloo.ca}} {$\mid$}
%     \href{mailto:4guhaniyer@gmail.com}{\raisebox{-0.2\height}{\faEnvelope[regular]}\ \underline{4guhaniyer@gmail.com}} {$\mid$}
%     \href{https://linkedin.com/in/guhansiyer/}{\raisebox{-0.2\height}\faLinkedin\ \underline{guhansiyer}} {$\mid$} 
%     \href{https://github.com/guhansiyer}{\raisebox{-0.2\height}\faGithub\ \underline{guhansiyer}}
% \end{center}

%-----------EDUCATION-----------
\section{Education}
  \resumeSubHeadingListStart
    \resumeSubheading
      {University of Waterloo}{Expected Graduation: \textbf{April 2028}}
      {Bachelor of Applied Science in Computer Engineering}{Waterloo, Ontario}
      \resumeItemListStart
        \resumeItem{Relevant Coursework: Real-Time Operating Systems, Algorithms \& Data Structures, Digital Computers}
      \resumeItemListEnd
  \resumeSubHeadingListEnd
  
%-----------WORK EXPERIENCE-----------
\section{Experience}
  \resumeSubHeadingListStart

    \resumeSubheading
      {Nokia}{Sept. 2025 -- Dec. 2025}
      {Firmware Engineering Intern}{Ottawa, Ontario}
      \resumeItemListStart
        \resumeItem{Built \textbf{C++} device initialization service for a new optical transceiver ASIC, optimizing boot sequence control across hardware modules.} % add some figure about latency improvement/performance optimzations: "optimizating boot sequences and reducing X by Y%/sec/."
        \resumeItem{Eliminated packet corruption in a critical \textbf{C++} message-passing utility with 128-bit atomic operations, \textbf{resolving race conditions} and ensuring data integrity.}
        \resumeItem{Resolved \textbf{10+} major defects in the ASIC SDK by analyzing firmware trace logs, reducing daily crash frequency to \textbf{zero}.}
      \resumeItemListEnd
      
    \resumeSubheading
      {Ford Motor Company}{Jan. 2025 -- Apr. 2025}
      {Software Development Intern}{Waterloo, Ontario}
      \resumeItemListStart
        \resumeItem{Created a modular library in \textbf{Python} to simplify and scale testing for an in-vehicle security daemon, reducing reliance on external tooling.}
        \resumeItem{Refactored a deprecated generator utility to integrate with new \textbf{Python} test infrastructure, enabling and enhancing new testing workflows for \textbf{100+ engineers}.}
        \resumeItem{Reworked \textbf{30+} legacy tests to utilize the new library, standardizing test structure for \textbf{all downstream developers}.}
      \resumeItemListEnd
      
    \resumeSubheading
      {NCR Voyix}{May 2024 -- Aug. 2024}
      {Software Engineering Intern}{Waterloo, Ontario}
      \resumeItemListStart
        \resumeItem{Integrated an internal \textbf{Python} query utility into a new patch management system, enabling automatic device data retrieval.}
        \resumeItem{Instrumented a service to validate per-device network compliance data for use \textbf{organization-wide}.}
        \resumeItem{Developed a cross-platform patch verification tool serving \textbf{10,000+} devices across \textbf{10+} platforms.}
      \resumeItemListEnd
    \resumeSubHeadingListEnd
    
%-------------------------------------------
%-----------PROJECTS-----------
\section{Projects}
    \resumeSubHeadingListStart

    % \resumeProjectHeading
    %       {\href{}{ \textbf{ARM Microkernel}}  \space  $|$ \textit{C, Raspberry Pi, ARM Assembly}}{November 2025}
    %       \resumeItemListStart
    %         \resumeItem{Designed core kernel functionality including interrupt vectors, inter-process communication, and memory-mapped device I/O.}
    %         \resumeItem{Implemented a \textbf{cooperative multitasking scheduler} with low-overhead context switching for lightweight user-process management.}
    %         \resumeItem{Built a custom \textbf{ARM} bootloader to initialize hardware peripherals for a stable system boot sequence.}

    %     \resumeItemListEnd

    \resumeProjectHeading
      {\href{https://github.com/WilliamZhang20/ECE298A-TPU}{ \textbf{TT-TPU} \faIcon{external-link-alt}}  \space  $|$ \textit{Verilog, Python}}{}
      \resumeItemListStart
        \resumeItem{Designed a simplified tensor processing unit supporting \textbf{2-by-2 integer matrix multiplication}; submitted to \textbf{Tiny Tapeout} project for manufacturing.}
        \resumeItem{Implemented an on-chip scratchpad memory in \textbf{Verilog} to stage input and weight matrices, enabling continuous streaming to the compute datapath and sustaining \textbf{$\sim$99.8 MOP/s} of throughput.}
        \resumeItem{Verified designs with \textbf{Python}/\textbf{cocotb} and ran static timing analysis with \textbf{OpenSTA}, ensuring functional correctness.}
    \resumeItemListEnd

    \resumeProjectHeading
          {\href{https://github.com/guhansiyer/wintop}{\textbf{wintop} \faIcon{external-link-alt}}  \space  $|$ \textit{C, MSVC, Windows API}}{}
          \resumeItemListStart
            \resumeItem{Developed a CLI-based \textbf{thread and process inspector} in \textbf{C} for \textbf{Windows} platforms, exposing detailed per-thread scheduling and runtime information.}
            \resumeItem{Enumerated processes and threads with \textbf{Win32 APIs}, retrieving priority, state, and timing metadata with low overhead.}
            \resumeItem{Designed a terminal interface to provide \textbf{real-time} diagnostic information, emulating \textit{top} and \textit{ps}.}
        \resumeItemListEnd

    \resumeProjectHeading
          {\href{https://github.com/guhansiyer/osh}{\textbf{osh: The Open Shell} \faIcon{external-link-alt}}  \space  $|$ \textit{C, Linux}}{}
          \resumeItemListStart
            \resumeItem{Created a lightweight \textbf{Linux} shell in C supporting built-ins and external programs.}
            \resumeItem{Implemented pipelines and I/O redirection using \textbf{Linux} syscalls and \textbf{POSIX} file descriptor semantics.}
            \resumeItem{Added persistent command history using \textbf{readline} to improve interactive usability.}
        \resumeItemListEnd
    \resumeSubHeadingListEnd

%-----------SKILLS-----------
\section{Skills}
 \begin{itemize}[leftmargin=0.15in, label={}]
    \small{\item{
                 \textbf{Languages}{: C, C++, Python, Java, Bash, Assembly (ARM, RISC-V)} \\
                 \textbf{Libraries \& Tools}{: Valgrind, GDB, CMake, Make, Android Tools (adb, Fastboot), Docker} \\
                 \textbf{Technologies \& Protocols}{: Linux, QNX, FreeRTOS, CAN, TCP/IP, UART, I2C, gRPC, protobuf}
            }
        }
 \end{itemize}

\end{document}